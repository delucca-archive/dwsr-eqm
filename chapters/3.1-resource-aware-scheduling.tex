\subsection{Resource-aware scheduling}
\label{subsec:resource-aware-scheduling}

Pupykina et al.~\cite{pupykina2019} presets a nice overview of the memory management field until 2019.
According to the authors, the main challenge on predicting memory usage is the lack of knowledge of the memory access patterns within an application.
Most of current research focus on using machine learning techniques to bypass this problem.
Although this approach has been successful, it is only capable of predicting the overall resource usage of a whole workload, and not the resource usage of a single task.

The observation made by Pupykina et al.~\cite{pupykina2019} is visible while evaluating recent work.
Most of the research done so far focus only on the scheduler perspective, using memory usage history to predict the expected resource requirements of a given cluster.

E. R. Rodrigues et al.~\cite{rodrigues2016} presents a machine learning model that can easily be integrated into a scheduler to predict the resource requirements of a given task.
Their work is mainly focused on the scheduler perspective, since it uses past executions by the user to predict the resource requirements for future jobs.
While submiting a new job to the cluster, the user must provide a manual estimate.
This estimate is used by E. R. Rodrigues et al.~\cite{rodrigues2016} alongside with past executions to predict the actual resource requirements of the job.
Although effective, this approach is vulnerable to spurious correlations, since past executions from other jobs by the same user can influence the prediction.

T. Mehmood et al.~\cite{mehmood2018} presents an ensemble machine learning model that can predict the expected resource usage of a cloud provider.
Like E. R. Rodrigues et al.~\cite{rodrigues2016}, T. Mehmood et al.~\cite{mehmood2018} uses past executions to predict the resource requirements at a given time.

% Althoug 
% B. T. Shealy et al.~\cite{shealy2021} presents a machine learning model that can consider both the memory usage history, as well as which run command flags were used to execute a given task, to predict the resource requirements of a given task.
% Their work is mainly focused on recurrent tasks, 
