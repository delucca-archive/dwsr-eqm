\subsection{Research questions and methodology}
\label{subsec:research-questions-and-methodology}

This section presents the research questions to explore in this study.
It also describes the methodology used to answer those questions.
Each research question is going be answered by a set of experiments, grouped into three different categories, including:
(i) feasibility,
(ii) accuracy,
(iii) and applicability.
Table~\ref{tab:research-questions} summarizes the research questions and their categories.

\begin{table}[ht]
  \caption{Research questions}
  \label{tab:research-questions}
  \resizebox{\textwidth}{!}{%
    \begin{tabular}{@{}|l|l|l|@{} }
      \toprule
      \multicolumn{1}{|c|}{\textbf{\#}} & \textbf{Question}                                                            & \textbf{Category} \\ \midrule
      RQ1                               & How Dask~\cite{dask} deals with automatic chunking?                          & Feasibility       \\ \midrule
      RQ2                               & What is the optimal way of gathering memory-usage data?                      & Feasibility       \\ \midrule
      RQ3                               & Are seismic tensorial algorithms memory bounded by the \ac{CPU} or \ac{GPU}? & Feasibility       \\ \midrule
      RQ4                               & Which features do we need to extract from the input data?                    & Feasibility       \\ \midrule
      RQ5                               & What is the memory-usage behavior of our algorithms?                         & Accuracy          \\ \midrule
      RQ6                               & Under extreme circumstances, how is the memory usage of our algorithms?      & Accuracy          \\ \midrule
      RQ7                               & How to integrate the reinforcement learning model to execute chunking?        & Applicability     \\ \bottomrule
    \end{tabular}
  }
\end{table}

\subsubsection{Feasibility}
\label{subsubsec:feasibility-experiments}

This experiment category aims to answer \textbf{RQ1}, \textbf{RQ2}, \textbf{RQ3}, and \textbf{RQ4}.
The goal is to understand the feasibility of our seismic operators and their main characteristics.

This category will start with an experiment to explore how Dask~\cite{dask} deals with automatic chunking (RQ1).
Then, it will proceed to figure out the proper way to gather memory usage metrics from both the \ac{CPU} and \ac{GPU} (RQ2).
After this, the following experiment will explore if our seismic operators are \ac{GPU}-memory bounded or \ac{CPU} memory bounded (RQ3).
Lastly, the final experiment will explore multiple executions of seismic operators, trying to understand possible features from the input data (RQ4).

\subsubsection{Accuracy}
\label{subsubsec:accuracy-experiments}

This experiment category aims to answer \textbf{RQ5} and \textbf{RQ6}.
The goal is to understand how reliable the results can be.
It will start by exploring the behavior of the seismic operators and how they use memory in different synthetic executions (RQ5).
Finally, it will try to push the existing algorithms to the limit to check if, under extreme circumstances (like uncommon inputs), the algorithm can break or display an unpredictable memory-usage pattern (RQ6).

\subsubsection{Applicability}
\label{subsubsec:applicability-experiments}

The third and last experiment category is applicability, which aims to answer RQ7.
The goal is to act as a pre-prototype experiment.
This phase will explore reinforcement learning models to understand how to apply the results from the previous experiments to adjust the block size based on the algorithm's memory footprint.

\subsubsection{Prototype development}

After the execution of all the experiments, the next step is implementing a prototype.
The idea for that prototype is to act as a plugin for \ac{DASF}~\cite{dasf} and use it to contribute to the active Petrobras seismic project in the laboratory.
The plugin can act to tune the input block\_size for every operator and as a decision heuristic for scheduling.

\subsubsection{Research extension}

This part explores the possibility of generalizing the developed machine-learning model to be algorithm-agnostic.
The decision-making process to explore this alternative will consider the result of all past experiments.
The first step to implementing this is understanding how and which features to extract from the source code.
The existing model will be improved to act as a generic model for any algorithm.
