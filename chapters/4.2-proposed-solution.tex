\subsection{Proposed solution}
\label{subsec:proposed-solution}

The proposed solution is divided into two stages:
(i) predicting the memory-footprint of an algorithm, and
(ii) automating the data partitioning process.
Each stage is presented as a subsection of this section.

\subsubsection{Predicting the memory-footprint of an algorithm}

Most seismic operators are tensorial algorithms.
Due to this fact, we can assume that the memory footprint of an algorithm is proportional to the shape of the input data.
Based on this, to predict the memory-footprint of an algorhtm I plan to create a machine learning model that would predict the memory usage of an algorithm based on the input data. 

The proposed solution can be executed in two different ways:

\begin{enumerate}
  \item \textbf{Algorithm-specific model:} train a model for each algorithm, considering the seismic data's shape and size as the primary features for the prediction.
  \item \textbf{Generic algorithm model:} train a model that is algorithm-agnostic, considering the source code as also a feature for the prediction.
\end{enumerate}

Although creating a generic algorithm model is more flexible, I plan to start by coding an algorithm-specific model to understand the problem better.
While coding the model, I expect to find input features that are relevant tot he prediction.
My initial hypothesis is that the input data's shape and size are the most relevant features, but I will experiment with other features to improve the prediction too.

Depending on the results of the initial experiments to create an algorithm-specific model, I may pursue the generic algorithm model.
This secondary goal is to develop a model that is algorithm-agnostic.
I understand that the source code of the algorithm contais relevant information, such as how the code author deals with memory management.
However, I do not have a clear picture of the features that I could extract from the source code, but I believe that this is a possible experiment for this phase of the research.

\subsubsection{Automating the data partitioning process}

The second stage of the proposed solution is to automate the data partitioning process.
The goal is to create a DASF~\cite{dasf} plugin that can automatically set the optimal block\_size parameter during execution based on a machine learning model that can predict the memory-footprint of the algorithm.
The plugin would be executed before the algorithm starts, and it would set the block\_size parameter based on the prediction of the model.
