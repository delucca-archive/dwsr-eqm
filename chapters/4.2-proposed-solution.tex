\subsection{Proposed solution}
\label{subsec:proposed-solution}

I see two possible solutions to optmize the block\_size parameter:
(i) I can create a model that would predict the ideal block\_size of algorithm based on the input data or
(ii) I can create a model that would predict the memory usage of an algorithm based on the input data, and use it to determine the optimal block\_size.

Although appproach (i) is more straightforward, it is limited only to the block\_size.
Approach (ii), on the other hand, is broader and can be used not only to identify the block\_size, but also as an heuristic for other tasks like scheduling, resource allocation, and cloud cost estimates.

Based on this, I have decided to pursue approach (ii) and create a machine learning model that would predict the memory usage of an algorithm based on the input data. 
While there are already alternatives available to predict memory consumption from the scheduler's perspective, no research has been conducted to predict memory consumption from an algorithmic perspective.

The proposed solution is divided into two strategies:

\subsubsection{Strategy one: Algorithm-specific model}
\label{subsubsec:algorithm-specific-model}

The first and primary goal of this research is creating a machine learning model that can predict the memory consumption of a given algorithm given the input data.
During this strategy, I will train a model for each algorithm, considering the seismic data's shape and size as the primary features for the prediction.
Since seismic dataset are usually normalized, I may also extract other features to improve the prediction, but such features are going to be defined during the exploration phase.
I will start the exploration with polynomial model, since I belive it is suitable for the prediction, and the model's output would help us determine the optimal block\_size for each part of the graph.

This approach's primary limitation is that it requires a new model for each seismic operator, but it allows us to predict the output size and shape, enabling us to compose a graph prediction.

\subsubsection{Strategy two: Generic algorithm model}
\label{subsubsec:generic-algorithm-model}

Our secondary goal is to develop a model that is algorithm-agnostic.
If the results of the first approach are optimistic, I may extract features from the source code itself to predict memory consumption accurately.
I believe that the algorithm could extract relevant information, such as how the code author deals with memory management.
Although I do not have a clear picture of the features that I could extract, I believe that this is a possible experiment for this research.
However, this objective is secondary and will be pursued only if the results of the first approach are positive.
