\subsection{Machine Learning}
\label{subsec:ml}

Machine learning is a subset of \ac{AI} that focuses on the development of algorithms that enable computers to learn from and make predictions or decisions based on data.
Instead of relying on explicit programming, machine learning systems learn patterns and relationships within the data, which allows them to adapt and improve their performance over time.
This powerful technology has rapidly become an integral part of various industries, revolutionizing the way we approach tasks and solve complex problems.

Machine learning can be broadly categorized into three types: supervised learning, unsupervised learning, and reinforcement learning.

\begin{itemize}
    \item \textbf{Supervised Learning:} In this approach, the machine learning model is trained on a labeled dataset, which consists of input-output pairs.
        The model learns the underlying relationship between the input and output data by minimizing the error in its predictions.
        This method is commonly used for tasks such as classification, where the goal is to categorize input data into predefined classes, and regression, where the goal is to predict a continuous value;
    \item \textbf{Unsupervised Learning:} Unlike supervised learning, unsupervised learning algorithms work with datasets that do not have labels.
        The goal is to discover hidden patterns, structures, or relationships within the data.
        This type of learning is often used for tasks such as clustering, where the aim is to group similar data points together, and dimensionality reduction, where the goal is to reduce the number of features while preserving essential information;
    \item \textbf{Reinforcement Learning:} In reinforcement learning, the model, known as an agent, learns to make decisions based on the consequences of its actions in a given environment.
        The agent receives rewards or penalties depending on the actions it takes, which helps it to learn an optimal policy over time.
        Reinforcement learning is widely used in robotics, gaming, and various control systems.
\end{itemize}

Machine learning has transformed various industries and sectors by providing innovative solutions to complex problems.
Some of the key issues it solves include:

\begin{itemize}
    \item \textbf{Data Analysis and Predictive Analytics:} Machine learning has revolutionized the way businesses analyze vast amounts of data.
        By automating the process of extracting insights from data, it enables organizations to make better-informed decisions and predict future trends more accurately.
        This can lead to improved operational efficiency, targeted marketing strategies, and enhanced customer experiences;
    \item \textbf{\ac{NLP}:} Machine learning has significantly improved our ability to process and understand human language.
        This has led to the development of advanced NLP applications, such as chatbots, sentiment analysis, and machine translation, which can understand, generate, and translate text in a manner similar to humans;
    \item \textbf{Image and Speech Recognition:} Machine learning has enabled computers to accurately recognize images and speech, paving the way for applications such as facial recognition systems, voice assistants, and autonomous vehicles.
        These systems rely on deep learning, a subset of machine learning that utilizes artificial neural networks to learn complex patterns in data;
    \item \textbf{Anomaly Detection:} Machine learning algorithms can efficiently detect unusual patterns or outliers in large datasets, making them valuable tools for anomaly detection.
        This is particularly useful in areas such as cybersecurity, where machine learning can help identify potential security threats, and in finance, where it can detect fraudulent transactions;
    \item \textbf{Personalization:} Machine learning enables the delivery of personalized experiences to users based on their preferences and behavior.
        This has led to the development of recommender systems, which provide personalized suggestions for products, movies, or news articles, improving user engagement and satisfaction.
\end{itemize}

Despite the immense potential of machine learning, there are challenges and limitations to consider, including data quality, algorithmic bias, and interpretability.
Ensuring that the data used to train machine learning models is accurate, diverse, and representative is essential for reliable predictions.
Additionally, understanding and addressing the biases that may be present in the data
