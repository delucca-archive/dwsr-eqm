\subsection{Memory footprint}
\label{subsec:memory-footprint}

The memory footprint of an algorithm refers to the amount of memory space required for its execution, considering both static and dynamic memory allocations.
It is a crucial performance metric, as it directly affects the overall system resources and impacts the efficiency and scalability of an algorithm.
In essence, the memory footprint is an essential aspect of an algorithm's resource consumption, alongside other factors such as time complexity and processing power.

\emph{Static memory allocation} is the memory allocated during compile-time, which includes the memory needed for storing executable code, global variables, and static local variables.
This memory remains fixed throughout the program's execution and is typically allocated in the text, data, and \ac{BSS} segments.

On the other hand, \emph{dynamic memory allocation} refers to the memory allocated during run-time, including the memory needed for storing dynamically allocated variables, function call stacks, and memory required by recursive functions.
This type of memory allocation occurs in the heap and stack segments.

To achieve a deeper understanding of the memory footprint of an algorithm, it is possible to evaluate the following components:

\begin{itemize}
    \item \emph{Space Complexity:} Measures the total amount of memory an algorithm uses as a function of its input size.
        This evaluation usually uses the big-O notation, such as $O(n)$ or $O(n^2)$, where $n$ is the input size.
        It is possible to split space complexity into two subtypes:
        (i) auxiliary space (temporary memory used during execution) and
        (ii) input space (memory needed for storing input data);
    \item \emph{Data Structures:} An algorithm's choice of data structures significantly influences its memory footprint.
        Different data structures have varying memory overheads and trade-offs, and selecting the most appropriate data structure can lead to substantial improvements in memory usage;
    \item \emph{Memory Management Techniques:} The way an algorithm manages memory can substantially impact its memory footprint.
        This includes memory allocation and deallocation strategies, garbage collection, and other techniques that optimize memory usage during the algorithm's execution.
\end{itemize}

The memory footprint of an algorithm is a critical aspect of its performance, as it directly influences the overall system resources and efficiency.
By comprehending and optimizing the memory footprint of an algorithm, developers can create more efficient and scalable solutions, ultimately contributing to improved computational capabilities in various applications and industries.
