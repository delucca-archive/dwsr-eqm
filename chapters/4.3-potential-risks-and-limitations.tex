\subsection{Potential risks and limitations}
\label{subsec:potential-risks-and-limitations}

In this section, I will discuss the limitations of the proposed solution for predicting memory consumption of seismic algorithms using machine learning.

\subsubsection{Sensibility towards too many control statements}

A possible limitation for this solution is that it may not work if the algorithm has too many control statements.
Depending on the structure of the algorithm, the amount of control statements may change the memory usage.

This limitation can be ignored if the memory allocation of the algorithm happens before the control statements.
On this scenario, even with control statements that change drastically the execution of the algorithm the memory usage will not change.

Either way, all the seismic operators and machine learning models being used are tensorial algorithms, which means that they do not have too many control statements.
Therefore, this possible limitation will not be a problem during the research.

\subsubsection{Unpredictable bottlenecks}

There are two possible memory bottlenecks in the proposed solution: CPU and GPU memory.
Seismic operators are likely to be GPU-memory bounded, but I need to conduct experiments to verify this.

If the algorithms can be both GPU-memory and CPU-memory bounded, then the proposed solution must be able to handle both scenarios.

\subsubsection{Language-agnosticity}

The proposed solution is currently focused on using Python since it is the language used for the seismic operators and DASK~\cite{dask} itself.

Although Python is a popular language in the scientific community, it may not be the language of choice for all researchers.
This solution may not be language-agnostic at the moment.
However, in the future, I can improve the solution to accept algorithms from any language.

In the case of strategy one, presented on section ~\ref{subsubsec:algorithm-specific-model}, I can improve the training structure to accept algorithms from any language.
This can be done by allowing decoupling both the feature extraction, as well as the execution of the algorithm, from the training process.

In the case of strategy two, presented on section~\ref{subsubsec:generic-algorithm-model}, it is possible improve the feature-extraction part to allow more languages as well.
Since the only difference between the two strategies is allowing to exract features from the source code, a specialized feature extractor per-language should be enough.


